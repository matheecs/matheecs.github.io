% !TEX TS-program = xelatex
% !TEX encoding = UTF-8 Unicode
% !Mode:: "TeX:UTF-8"

\documentclass{resume}
\usepackage{geometry}
\geometry{top=0.7cm,bottom=0.6cm,left=1.5cm,right=1.5cm}
\usepackage{amsmath}
\usepackage{bm}

% \usepackage{zh_CN-Adobefonts_external}
\usepackage{xeCJK}
\usepackage{linespacing_fix}
\usepackage{romannum}

\begin{document}
\pagenumbering{gobble}

\name{吉祥}

\basicInfo{
  \email{matheecs@qq.com} \textperiodcentered\
  \phone{(+86) 188838912\{41+8\}} \textperiodcentered\
  \homepage{https://matheecs.tech}
}

\section{\faGraduationCap\  教育背景}
\datedsubsection{\textbf{西安交通大学}}{2017.9$\rightarrow$2020.6}
{硕士}\ $\mathbf{GPA=3.79/4.00}$, 控制科学与工程
\datedsubsection{\textbf{University of Wisconsin-Milwaukee}}{2017.1$\rightarrow$2017.5}
优秀本科生国际交流项目%重庆大学美国威斯康辛大学密尔沃基分校电气工程专业
\datedsubsection{\textbf{重庆大学}}{2013.9$\rightarrow$2017.6}
{学士}\ $\mathbf{GPA=3.57/4.00}$, 电气工程及其自动化

\section{\faUsers\ 工作经历}
\datedsubsection{\textbf{UR, 机器人工程师}}{2022.\Romannum{10}$\rightarrow$Present}
\begin{onehalfspacing}
\begin{itemize}[parsep=0.5ex]
  \item Model Predictive Control
  \item Whole-Body Control
  \item Control Framework Design
\end{itemize}
\end{onehalfspacing}

\datedsubsection{\textbf{云深处科技, 机器人规划算法工程师}}{2020.7$\rightarrow$2022.9}
\begin{onehalfspacing}
负责四足机器人自主巡检业务的任务决策、路径规划与轨迹优化模块;研究物理引擎仿真与多刚体动力学算法;开发四足机器人基于MPC的运动控制算法与基于直接配点法的轨迹优化算法;开发用于四足机器人自主导航的实时楼梯检测算法;研发基于因子图的机械臂手眼标定算法。
\end{onehalfspacing}

\datedsubsection{\textbf{旷视科技, 研究院SLAM组实习生}}{2019.6$\rightarrow$2019.9}
\begin{onehalfspacing}
负责仓库机器人的稀疏点云地图构建模块,提升室内\textbf{视觉重定位}精度,采用深度学习提取SuperPoint特征、光流跟踪、多帧三角化方法在TX2计算平台实现在线建图,最终让重定位精度提升了50\%;研究基于图像的三维重建SfM算法、Visual Localization定位方法。
\end{onehalfspacing}

\datedsubsection{\textbf{ICRA 2019 DJI RoboMaster}}{2019.1$\rightarrow$2019.5}
\begin{onehalfspacing}
负责开发全自动射击对抗机器人的多机器人自主决策模块,采用 ROS 和 C\texttt{++} 设计行为树实现决策功能,用目标检测(灯带)与PnP定位敌方,根据场上形势自主决策、运动规划与控制,通过发射弹丸击打敌方机器人进行射击对抗。凭借出色的决策算法和系统鲁棒性在国内外60支队伍的较量中取得全球季军。
\end{onehalfspacing}
% \datedsubsection{\textbf{北京初速度科技(Momenta)有限公司, 足球机器人开发}}{2018.7$\rightarrow$2018.8}
%\role{hackRobot团队组长}{\textbf{北京初速度科技有限公司}}
% \begin{onehalfspacing}
% 担任团队组长带领团队开发全自主足球机器人,基于 TurtleBot3 移动平台、树莓派、单目相机、IMU和码盘传感器等硬件平台,开发基于深度学习的目标检测(门框)算法与PnP方法实现机器人的视觉重定位功能,最终取得团队亚军。
% \end{onehalfspacing}
% \datedsubsection{\textbf{视觉SLAM/VIO/SfM理论研究与工程实践}}{2018.1$\rightarrow$2019.9}
% %\role{优秀学员}{\textbf{深蓝学院}}
% \begin{onehalfspacing}
% 担任深蓝学院\textbf{从零开始手写VIO}课程助教。完成深蓝学院\textbf{视觉SLAM理论与实践}课程,被评为优秀学员(TOP 10\%)。
% %研读视觉SLAM领域的论文著作:{视觉SLAM十四讲}、{Multiple View Geometry in Computer Vision}、{State Estimation for Robotics},专研算法的工程实现,阅读S-MSCKF、ORB-SLAM2项目源代码,掌握COLMAP三维建模工具。
% \end{onehalfspacing}
% \datedsubsection{\textbf{中国大学生智能设计竞赛}}{2016.3$\rightarrow$2016.8}
%\role{团队组长}{\textbf{中国大学生智能设计竞赛}}
% \begin{onehalfspacing}
% 担任团队组长带领团队开发智能仓库机器人,负责设计仓库机器人的技术方案、设备采购与调试,采用TurtleBot2、ROS、Arduino、三自由度机械臂实现机器人的自主定位导航、目标识别与自主抓取的功能,最终带领团队在国内100多支队中取得全国一等奖。
% \end{onehalfspacing}
%\datedsubsection{\textbf{STEM创客培训项目}}{2014年}
%%\role{培训主讲}{\textbf{重庆大学}}
%\begin{onehalfspacing}
%%	\href{http://huxi.cqu.edu.cn/page/e8b514e32b9176cf}{作为项目发起人,负责培训项目的策划、申请、宣传组织和培训演讲}
%本科期间第一个由自己构思、组织实施的创客培训项目,培训内容为电子设计和Arduino入门。
%\end{onehalfspacing}
%\datedsubsection{\textbf{个人作品}}{}
%\role{团队组长}{\textbf{大学生科研训练计划项目}}
%\begin{onehalfspacing}
%\href{http://www.arduino.cn/thread-17552-1-1.html}{采用Arduino开源平台,负责软硬件系统设计}。采用MIT App Inventor实现上位机应用开发,用C\texttt{++}语言实现自主循迹模式与手机遥控模式
%{涵道式三维重建巡检机器人},{仿Genghis六足机器人}。
%\end{onehalfspacing}
% Reference Test
%\datedsubsection{\textbf{Paper Title\cite{zaharia2012resilient}}}{May. 2015}
%An xxx optimized for xxx\cite{verma2015large}
%\begin{itemize}
%  \item main contribution
%\end{itemize}
\section{\faCogs\ 综合能力}
% increase linespacing [parsep=0.5ex]
\begin{itemize}[parsep=0.5ex]
  \item 理论基础: 线性代数,刚体动力学,最优控制,数值优化%{GAMES103|104|105}
  \item 开发工具: C\texttt{++}/Python,CMake,Eigen,\{Pinocchio,MuJoCo,OSQP\}
\end{itemize}

\section{\faHeartO\ 个人荣誉}
\datedline{ICRA 2019 RoboMaster 人工智能挑战赛\textbf{全球季军}}{2019年}
%\datedline{华为嵌入式精英挑战赛, \textbf{西北赛区一等奖}}{2019年}
%\datedline{第三期 Momenta 无人车足球赛, \textbf{团队亚军}}{2018年}
%\datedline{第六届“华为杯”中国大学生智能设计竞赛, \textbf{全国一等奖}}{2016年}
%\datedline{重庆大学大学生科研训练计划, \textbf{校级一等奖}}{2015年}
\datedline{\textbf{国家奖学金}}{2014年}

\end{document}
